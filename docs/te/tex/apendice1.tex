\section{Algoritmo Principal:}
\label{Apendice1:XXX}

\begin{center}
\begin{footnotesize}
\begin{verbatim}
###################################################################################
# sol_geo.py
###################################################################################
#
# AUTORES: Maria Baeza Lopez
#          Juan Jesus Doniz Labrador
#          Jesus Manuel Rodriguez Falcon
#
# FECHA: 09/05/2014
#
# DESCRIPCION: Este programa realiza el calculo de la Distribucion Geometrica, dados los
#			   parametros de entrada n y p, utilizando el modulo: mod_geo.py.
#              Este programa da 4 opciones al usuario para el calculo de dicha distribucion:
#			   (0=P(X=n),1=P(X<=n,2=P(X<n),3=P(x>=n))
#
###################################################################################

#!/usr/bin/python
#!encoding: UTF-8
import sys
import mod_geo
import time
import timeit
# Menu principal
argumentos = sys.argv[1:]
 # print argumentos         Imprime la lista con los parametros que le des desde la consola
if (len(argumentos) == 2):# si la lista es de dos elementos (n,p)
  n = int(argumentos[0])
  p = float(argumentos[1])
else:
    print "Introduzca el n de pruebas necesarias para obtener un exito (n>0):"
    n = int (raw_input())
    print "Introduzca el valor p (p>0):"
    p = float(raw_input())
if (n > 0):
  print "¿Que tipo de probabilidad vas a hallar? (0=P(X=n), 1=P(X<=n), 2=P(X<n), 3=P(X>=n))"
  respuesta=int(raw_input())
  if (respuesta==0):
    start=time.time()
    probabilidad=mod_geo.calcular_geo (n,p)
    finish=time.time()-start
    print "La probabilidad P[X= %d] con p= %f es: %f y 
           he tardado %f segundos en calcularlo"  %(n,p,probabilidad,finish)
  elif (respuesta==1):
    start=time.time()
    probabilidad=mod_geo.calcular_geo1(n,p)
    finish=time.time()-start

    print "La probabilidad P[X<= %d] con p= %f es: %f y 
           he tardado %f segundos en calcularlo"  %(n,p,probabilidad,finish)
  elif (respuesta == 2):
    start=time.time()                        
    probabilidad=mod_geo.calcular_geo1(n-1,p)
    finish=time.time()-start                 

    print "La probabilidad P[X< %d] con p= %f es: %f y 
           he tardado %f segundos en calcularlo"  %(n,p,probabilidad,finish)
  else:
    start=time.time()                       
    prob=mod_geo.calcular_geo1(n,p)
    probabilidad = 1 - prob
    finish=time.time()-start                 

    print "La probabilidad P[X>= %d] con p= %f es: %f y 
           he tardado %f segundos en calcularlo"  %(n,p,probabilidad,finish)

else:
    print "No podemos hallar la probabilidad"


\end{verbatim}
\end{footnotesize}
\end{center}

\newpage
\section{Algoritmo del Modulo:}
\label{Apendice1:YYY}

\begin{center}
\begin{footnotesize}
\begin{verbatim}
###################################################################################
# mod_geo.py
###################################################################################
#
# AUTORES: Maria Baeza Lopez
#          Juan Jesus Doniz Labrador
#          Jesus Manuel Rodriguez Falcon
#
# FECHA: 09/05/2014
#
# DESCRIPCION: Este modulo contiene dos funciones:
#             1. calcular_geo (n,p)---->calcula la probabilidad geometrica
#			  y la devuelve a sol_geo.py
#             2. calcular_geo1 (n,p)--->Calcula la probabilidad geometrica
#             desde 1 hasta n y las va almacenando en sumaprobabilidad
#
###################################################################################

#!/usr/bin/python
#!encoding: UTF-8
import sys
import math

def calcular_geo (n,p):
  if (n>1):
    q=1-p
    probabilidad=(q**(n-1))*p
  else:
    probabilidad=p
  return (probabilidad)

def calcular_geo1 (n,p):
  probabilidad=0
  for i in range (1,n+1):
    sumaprobabilidad=calcular_geo(i,p)
    probabilidad+=sumaprobabilidad
  return (probabilidad)
\end{verbatim}
\end{footnotesize}
\end{center}
