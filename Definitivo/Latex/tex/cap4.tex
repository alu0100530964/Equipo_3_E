%%%%%%%%%%%%%%%%%%%%%%%%%%%%%%%%%%%%%%%%%%%%%%%%%%%%%%%%%%%%%%%%%%%%%%%%%%%%%
% Chapter 4: Conclusiones y Trabajos Futuros
%%%%%%%%%%%%%%%%%%%%%%%%%%%%%%%%%%%%%%%%%%%%%%%%%%%%%%%%%%%%%%%%%%%%%%%%%%%%%%%
Tras el estudio y ejecuci\'on de varios tipos de problemas sobre la Distribuci\'on Geom\'etrica, se puede observar que:\\
\begin{enumerate}
	 \item Se nos asign\'o en este trabajo  hacer un programa Python que solucionara la funci\'on de probabilidad de la distribuci\'on geom\'etrica, para ello, tuvimos que buscar informaci\'on sobre esta distribuci\'on (def calcular geo1).

\item Despu\'es iniciamos la implementaci\'on de la funci\'on de la probabilidad, para ello creamos una funci\'on (def calcular\_geo).
   Esta funci\'on nos era \'util en alguno de los problemas, pero no en otros casos. \'Este problema lo solucionamos mediante ensayo y error, resolviendo as\'i nuestro dilema incorporando una nueva funci\'on.
  \item Podemos sacar como conclusi\'on final que:
   \begin{enumerate}
   	 \item Debemos ejecutar varias veces un programa, hasta purificarlo y sacar un programa eficiente y operativo.

     \item Adem\'as, gracias a nuestra b\'usqueda de informaci\'on hemos podido refrescar y ampliar conocimientos sobre probabilidades, m\'as concretamente sobre la distribuci\'on geom\'etrica.
     \item Nos hemos dado cuenta del enorme potencial que tiene la utilizaci\'on de \LaTeX, Beamer y Python, para la elaboraci\'on documental, de presentaci\'on y de creaci\'on de algoritmos, respect\'ivamente. Nos ser\'a de gran ayuda en la elaboraci\'on de nuestros escritos, en el marco de nuestra formaci\'on acad\'emica universitaria y laboral.
   \end{enumerate}

\end{enumerate}




