\section{C\'odigo para generar gr\'aficas}
\label{Apendice2:label}

\begin{center}
\begin{footnotesize}

\begin{verbatim}
#!/usr/bin/python
#! enconding: UTF-8

import pylab as dibujo

x = [20,40,60,80,100]
y1 = [0.000051,0.000045,0.000046,0.000046,0.000048]
y2 = [0.000095,0.000112,0.000133,0.000158,0.000183]
y3 = [0.000086,0.000112,0.000134,0.000159,0.000185]
y4 = [0.000089,0.000113,0.000139,0.000164,0.000188]

dibujo.title ('Tiempo')

dibujo.plot (x, y1,linestyle='-.',marker='*',color='r', label='Para el caso 0')  #':' '--' '_' . 
Los colores son 'r'  junto a m b g c y.    Los marker son 'o' junto a s p * + . ^
dibujo.plot (x, y2,linestyle='--',marker='*',color='b', label='Para el caso 1')
dibujo.plot (x, y3,linestyle='-',marker='*',color='g', label='Para el caso 2')
dibujo.plot (x, y4,linestyle=':',marker='*',color='m', label='Para el caso 3')

dibujo.legend()      # Esto se pone para que aparezca la leyenda de las lineas
dibujo.xlim(10,155)
dibujo.ylim(0.000044,0.000190)
dibujo.xticks(x)
dibujo.xlabel ('Valores de la n')
dibujo.ylabel ('Tiempo en segundos')

dibujo.show()
\end{verbatim}

\end{footnotesize}
\end{center}
